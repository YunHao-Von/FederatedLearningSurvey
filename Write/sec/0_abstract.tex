\begin{abstract}
	Federated learning is a distributed machine learning paradigm that emerged as a solution to the need for privacy
	protection in artificial intelligence. Like traditional machine
	learning, federated learning is threatened by multiple attacks.  
	backdoor attacks,
	Byzantine attacks, and adversarial attacks. The weaknesses are
	exacerbated by the inaccessibility of data in federated learning,
	which makes it more difficult to defend against these threats.
	This points to the need for further research into defensive
	approaches to make federated learning a real solution for distributed machine learning paradigm with securing data privacy.
	This paper aims to enhance understanding of the threats faced by
	federated learning and their defense mechanisms, and assist the
	academic and industrial communities in developing more robust
	federated learning systems.
	Our survey provides a taxonomy of these threats and defense
	methods, describing the general situation of this vulnerability
	in federated learning. We also sort out
	the relationship between these methods, their advantages and
	disadvantages, and discuss future research
	directions regarding the security issues of federated learning
	from multiple perspectives.

\end{abstract}