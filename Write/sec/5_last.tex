\section{Advanced Research and Problems}
\label{Advanced Research and Problems}
\subsection{Stealthiness}

A good attack method should not be easily detected by
the defense system. Backdoor attackers need to effectively
hide backdoors to make them difficult to detect. In
Byzantine attacks, how do Byzantine attackers launch
attacks while evading detection is also to be solved.
Adversarial attacks need to generate adversarial samples
without being detected. The attack methods developed
above enhance their stealthiness by exploiting long-tail
distribution, federated learning distribution, and dynamic
generation techniques. Long tail distributed attacks and
distributed attacks will eventually be discovered by the
larger and larger model capacity, so how to develop a
hidden attack method that can effectively face the large
capacity model is a major problem in the current attack
method research.   

\subsection{Trade-Off}
There are many Trade-off issues in federated learning.
In federated learning, client data is distributed among
multiple participants. The protection of privacy is an
important reason why federated learning is so popular.
So it is an important challenge to ensure data privacy
while implementing defenses. Effective defense
mechanisms, such as fault-tolerant algorithms and majority
voting, are needed to counter Byzantine attacks. However,
these mechanisms need to strike a balance with factors
like privacy preservation and computational efficiency.
Adversarial training can enhance model robustness but
may compromise generalization capability. Balancing the
trade-off between adversarial robustness and
generalization is a problem that requires exploration and resolution.
Recently, some works have studied the latent connection
between adversarial attacks and backdoor attacks. For
example, Weng et al.~\cite{weng2020trade} empirically demonstrated that
defending against adversarial attacks via adversarial
training may increase the risks of backdoor attacks. Solving
these various Trade-off problems is also the focus of future
research in federated learning.

\subsection{Convergence}
In defense methods, in order to improve the robustness
of the model, existing methods often improve the training
process of the model, such as distillation or adversarial
training. These methods do play a good defense effect, but
also bring the model convergence problems. The distributive
nature of federated learning further aggravates the
problem of difficult convergence, because the distributive
nature may cause problems such as heterogeneous data
and unbalanced computing resources. In the methods
mentioned in this paper, some articles have improved these
problems in defense methods, such as using smoothing
to produce equivalent results, or using reweighting to
improve decision boundaries to solve the impact of data
heterogeneity. But in general, how to make the model
converge quickly is still a problem that needs to be studied
in the future.
\section{Conclusion}
In this survey, we systematically introduce state of the art 
threats in federated learning systems, which mainly
include byzantine attacks, backdoor attacks and adversarial
attacks. And we also detailed describe corresponding
defensive policies. We also introduce the advantages and
disadvantages of these methods and how they work, and
sort out the relationship between them. Additionally, we
discuss a number of open problems of current defense
methods, hoping them could help researchers identify and
solve issues more quickly in the area of robust federated
learning.
